\documentclass[11p]{article}
% Packages
\usepackage{amsmath}
\usepackage{graphicx}
\usepackage[swedish]{babel}
\usepackage[
 backend=biber,
 style=authoryear-ibid,
 sorting=ynt
]{biblatex}
\usepackage[utf8]{inputenc}
\usepackage[T1]{fontenc}
%Källor
\addbibresource{mall.bib}
\graphicspath{ {./images/} }

\title{PMmall \\ \small Fysik 1}
\author{Levi Högdal }
\date{\today}

\begin{document}

 \begin{titlepage}
  \begin{center}
   \vspace*{1cm}

   \Huge
   \textbf{Title}

   \vspace{0.5cm}
   \LARGE
   Subtitle

   \vspace{1.5cm}

   \textbf{Levi högdal}

   \vfill

   Ett PM om energiförsörjning \\
   Fysik 1

   \vspace{0.8cm}

   \includegraphics[width=0.4\textwidth]{NTI Gymnasiet_Symbol_print_svart.png}

   \Large
   Teknikprogrammet\\
   NTI Gymnasiet\\
   Umeå\\
   \today

  \end{center}
 \end{titlepage}
% Om arbetet är långt har det en innehållsförteckning, annars kan den utelämnas
 \tableofcontents
 \newpage
 \section{Disposition hos ett PM}
 Ett PM har den mest informella strukturen av de vetenskapliga texterna. Det är egentligen bara en sammanställning av kunskap men för att den ska bli lite lättare att ta sig an brukar det finnas en inledning där syfte och frågeställningar redovisas och en avslutning där du kan dra slutsatser. Rubrikerna kan döpas valfritt, speciellt de som finns i huvuddelen av texten beror på vad den handlar om. Se nedan för ett exempel.

 \section{Inledning}
 Beskriv varför detta ämne är intressant eller viktigt. Vad är syftet med texten?

Vattenkraftverk är ett förnyelsebart sätt att framställa energi.
 Till skillnad från kol eller olja som det bara finns en viss mängd av tills det är slut och det är praktiskt omöjligt att öka den mängden som existerar så är vattenkraft en kraftkälla som det praktiskt sätt finns en stor mängd av som aldrig kommer ta slut.
Det låter väldigt bra att minska beroendet på kol och olja men vad är ett vattenkraftverk och hur fungerar det.

 \subsection{frågeställningar}
 \begin{enumerate}
  \item Hur fungerar ett vattenkraftverk?
  \item Vilka miljöpåverkan har ett vattenkraftverk lokalt och globalt?
  \item Hur påverkar vattenkraft samhället (Ekonomi/politik/konflikter/m.m.) lokalt och globalt?
 \end{enumerate}

 \section{Resultat}
 Här kommer allt med massor av mer rubriker och underrubriker
 \subsection{Varrenkraft, så fungerar det}

 Ett vattenkraftverk fungerar genom att man omleda vatten genom turbiner som skapar elektricitet.
 Turbinerna används omvandla lägesenergi från vattnet till användbar elektricitet.
 Enligt \textcite{NE} så beror den genererade på fallhöjden och vattenföringen och med fallhöjden 1 meter och vattenföringen 1 m${^3}$/s så kan man förväntade en effekt av 9.8 kW som sen med vattenkraftverkets verkningsgrad kan man komma fram till den elektriska effekten.
Olika Turbiner används beroende på hur mycket fallhöjden i vattenkraftverket är.

 \subsection{Miljökonsekvenser av Vattenkraft}

 \subsection{Vattenkraft bidrar till att minska konflikter om oljetillgångar i världen}
 All elproduktion har någon mängd av miljöpåverkan. Dels krävs råvarorna för att producera

 \section{Slutsatser}
 Här kan du dra slutsatser eller sammanfatta ditt resultat

% Mer saker som du kan ha nytta av.

 \section{Referenser}
 Referenser i text kan skrivas på två sätt: Enligt \textcite{Jens} kan man använde två typer av referenser, inbäddade i texten eller efter ett fakta \parencite{Fraenkel}. Ett till test för att se hur det ser ut \parencite[sid 55]{fermi}.

 \section{Annat som kan vara bra att veta}
 Om du vill ha kodstil och få med alla tecken kan du använda verbatim. Då kan du skriva \verb|abcd!"#| utan problem...

 Citat skrivs mellan de konstiga symbolerna \verb|``| och \verb|''| för att de ska se bra ut ``se bra ut!''.
 \subsection{En underrubrik}
 \subsubsection{En underunderrubrik}
 \subsection{Ekvationer}
 Det är lätt att skriva matematik i \LaTeX


 \begin{equation}
  F = G \frac{M m}{r^2}
  \label{grav}
 \end{equation}

 Ekvation (\ref{grav}) känner ni igen...

 \subsection{figurer}
 Bilder placeras enklast på detta sätt. placeringen bestämmer \LaTeX och vi kan bara föreslå (h)är, (t)opp eller (b)otten. Ett utropstecken före tvingar lite mer men inte absolut. I bild \ref{varg} visas en varg

 \printbibliography

\end{document}

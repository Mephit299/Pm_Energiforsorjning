\documentclass[11p]{article}
% Packages
\usepackage{amsmath}
\usepackage{graphicx}
\usepackage[swedish]{babel}
\usepackage[
 backend=biber,
 style=authoryear-ibid,
 sorting=ynt
]{biblatex}
\usepackage[utf8]{inputenc}
\usepackage[T1]{fontenc}
%Källor
\addbibresource{mall.bib}
\graphicspath{ {./images/} }

\title{PMmall \\ \small Fysik 1}
\author{Levi Högdal }
\date{\today}

\begin{document}

 \begin{titlepage}
  \begin{center}
   \vspace*{1cm}

   \Huge
   \textbf{Title}

   \vspace{0.5cm}
   \LARGE
   Subtitle

   \vspace{1.5cm}

   \textbf{Levi högdal}

   \vfill

   Ett PM om energiförsörjning \\
   Fysik 1

   \vspace{0.8cm}

   \includegraphics[width=0.4\textwidth]{NTI Gymnasiet_Symbol_print_svart.png}

   \Large
   Teknikprogrammet\\
   NTI Gymnasiet\\
   Umeå\\
   \today

  \end{center}
 \end{titlepage}
% Om arbetet är långt har det en innehållsförteckning, annars kan den utelämnas
 \tableofcontents
 \newpage
 \section{Disposition hos ett PM}
 Ett PM har den mest informella strukturen av de vetenskapliga texterna. Det är egentligen bara en sammanställning av kunskap men för att den ska bli lite lättare att ta sig an brukar det finnas en inledning där syfte och frågeställningar redovisas och en avslutning där du kan dra slutsatser. Rubrikerna kan döpas valfritt, speciellt de som finns i huvuddelen av texten beror på vad den handlar om. Se nedan för ett exempel.

 \section{Inledning}
 Beskriv varför detta ämne är intressant eller viktigt. Vad är syftet med texten?

Vattenkraftverk är ett förnyelsebart sätt att framställa energi.
 Till skillnad från kol eller olja som det bara finns en viss mängd av tills det är slut och det är praktiskt omöjligt att öka den mängden som existerar så är vattenkraft en kraftkälla som det praktiskt sätt finns en stor mängd av som aldrig kommer ta slut.
Det låter väldigt bra att minska beroendet på kol och olja men vad är ett vattenkraftverk och hur fungerar det.

 \subsection{frågeställningar}
 \begin{enumerate}
  \item Hur fungerar ett vattenkraftverk?
  \item Miljökonsekvenser av vattenkraft.
  \item Hur påverkar Vattenkraft ekonomin och samhället?
 \end{enumerate}

 \section{Resultat}

 \subsection{Varrenkraft, så fungerar det}

 Ett vattenkraftverk fungerar genom att man omleda vatten genom turbiner som skapar elektricitet.
 Turbinerna används omvandla lägesenergi från vattnet till användbar elektricitet.
 Enligt \textcite{NE} så beror den genererade på fallhöjden och vattenföringen och med fallhöjden 1 meter och vattenföringen 1 m${^3}$/s så kan man förväntade en effekt av 9.8 kW som sen med vattenkraftverkets verkningsgrad kan man komma fram till den elektriska effekten.
Olika Turbiner används beroende på hur mycket fallhöjden i vattenkraftverket är.

 \subsection{Miljökonsekvenser av vattenkraft}

 All elproduktion har någon mängd av miljöpåverkan. Dels krävs råvarorna för att producera energikällan och sen pruduktonen själv tar och påverkar miljön.
 För att bygga ett vattenkraftverk bygger man dammar över normala vattendrag.
 Att bygga dom här dammarna stör naturen, fiskar och djur.
 Som \textcite{vattenfall} säger kan man underlätta påverkan på fiskar genom att bygga fisktrappor så fisk faktiskt kan ta sig runt dammen och fortsätt existera relativ normalt.
 Miljökostnaden för ett vattenkraftverket ligger mest i dess konstruktion och underlining av anläggningen vilket fortfarande är mycket bättre en kol och olja.

 \subsection{Hur påverkar vattenkraft ekonomin och samhället?}
 Till samhället är hela funktionen av ett vattenkraftverk att få elektricitet som samhället kan använda.
 Det gör att samhället behöver vara mindre beroende på andra former av energi som kol, olja och naturgas.
 Vattenkraft går också att justera mängden elektricitet som produceras beroende på hur mycket el som används.
 Kol och olja kan också ersättas med vattenkraft vilket om det görs minskar kolutsläp och påverkan på global uppvärmning.


 \section{Slutsatser}
 Vattenkraftverk är relativt väldigt bra för miljön och gör mycket för samhället.
 Det är också väldigt stabil elkälla som går att öka och sänka mängden ström produceras efter vad marknaden kräver och kräver bara uppehåll för att det ska produceras el.
 Lokalt kan det göra så omgivningen ser sämre ut men i det hela är hur effektivt ett vattenkraftverk är och hur stort problem global uppvärmning är så blir all förnyelsebar energi en bra ide.


 \printbibliography

\end{document}
